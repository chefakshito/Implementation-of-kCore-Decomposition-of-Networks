\begin{itemize} 
\item{A general description:
Suppose we have a network of countries with trade deals between them represented as a graph. An edge between country A and country B represents a trade deal between the two countries. Now we want to find the sub-groups of countries with the maximal trade deals between them. We can run the k-core decomposition to find the sub-graph containing these countries. All the countries which have an edge between them and fall into the highest core can then be extracted as a layer. Now, say, we want to say the countries which have the maximal trade deals once the if the above countries had no trade deals between them. We will can then remove the edges between the above countries and run the k-core decomposition algorithm on the resulting graph again to find this new group of countries. The edges which fall into these groups are now the next layer of the network. 
} 

\item{User 1: } Learner
\item{Learner - Interaction Modes: }
Read page and play visualizations on given examples.
\item{Learner - Scenarios: }
	\begin{itemize} 
	\item{Scenario1 description: }
	A student is going to learn the k-core decomposition algorithm. The student can read the algorithm description and go through the interactive visualization of the algorithm on preset input. The student can then see how the size and density of the graph affects the computation time and concentration of edges in each layer of the graph and hence develop a better understanding of the algorithm.
	\item{System Data Input for Scenario 1: }
	The input is the selection of inputs given on the web site. 
    \item{Input Data Types for Scenario 1:}
    The pre-generated graphs can be selected from a list.
	\item{System Data Output for Scenario 1: }
	The website tool generates the visualization of the cores and subsequent layers of the graph.
    \item {Output Data Types for Scenario 1: }
    The output is a graph with the cores represented by colors as well as tool tips as layers generated in separate sections below the main graph.
    \item{Scenario 2 description:}
	An individual has seen or learned about the algorithm previously, but needs a quick refresher on how the algorithm works. They will just run a single preset and not fiddle with parameters.
	\item{System Data Input for Scenario 2: }
	The individual selects the graph they want to run it on.
    \item{Input Data Types for Scenario 2:}
    The user clicks on a preset on the preset list.
	\item{System Data Output for Scenario 2: }
    The individual sees the algorithm working over the graph and the resulting cores.
    \item {Output Data Types for Scenario 2: }
    Visualizations occur in UI elements on the webpage. The cores are represented by the colors and the layers are visualized. The computation time and edge concentration for each layer are also visualized.
	\end{itemize}
\item{User 2: } Teacher
\item{Teacher - Interaction Modes: }
Play visualizations on given examples and specialized examples which they input.
\item{Teacher - Scenarios: }
	\begin{itemize} 
	\item{Scenario 1 description:}
	A teacher would like an interactive, visual accompaniment to their lecture on the core decomposition. 
	\item{System Data Input for Scenario 1: }
	The teacher can select one of the preset graphs to run the algorithm on, or they can generate their own graphs as well with the accompanied graph generator program.
    \item{Input Data Types for Scenario 1:}
    The teacher can specify the number of nodes and edges desired in the graph to be processed to the random graph generator program. 
	\item{System Data Output for Scenario 1: }
	A random graph generated, which is then processed by the k-cores program and the resulting file is visualized using the web-tool.
    \item {Output Data Types for Scenario 1: }
    The graph generator outputs the graph as an adjacency list. The output from the k-cores algorithm is a JSON file which can be used with the web-tool.
    \item{Scenario 2 description:}
	The teacher does not have time to cover the algorithm in class, but wants students to learn it. They direct students to the web tool site for them to tinker with the algorithm themselves based on some graphs pre-generated by the teacher.
	\item{System Data Input for Scenario 2: }
	The teacher can specify specific inputs to run on or direct students to play around with example or even their own inputs.
    \item{Input Data Types for Scenario 2:}
    A special link to a web-page loaded with the teacher's desired presets generated by the graph generator program.
	\item{System Data Output for Scenario 2: }
	Each of the teacher's students see the visualization of the algorithm along with the k-cores in the graph and the subsequent layers extracted from the graph.
    \item {Output Data Types for Scenario 2: }
    Interactive visualizations along with graphs representing the computation time and edge concentration for each layer.
	\end{itemize}
\item{User 3: } A user who needs to analyze a network.
\item{User - Interaction Modes: }
Run the algorithm on specialized examples which they input. Download the resulting rankings to their computer.
\item{User - Scenarios: }
	\begin{itemize} 
	\item{Scenario 1 description:}
	An individual would like to quickly run the algorithm on a small dataset without having to write code or import a library.
	
	\item{System Data Input for Scenario1: }
	The individual would input the specific graph for which they want to run the algorithm on and will specify the graph explicitly.
    \item{Input Data Types for Scenario 1:}
    The user can input a raw text file containing the graph represented as an adjacency list.
	\item{System Data Output for Scenario 1: }
	
    \item {Output Data Types for Scenario 1:}
    The JSON files of the main graph and each layer which contain the distinct cores and can be run using the web-tool and the graph can be visualized and analyzed for the edge concentration in layers and computation time.
    \item{Scenario 2 description: Visualizing a particular layer of a very large graph}
	An individual would like to analyze only a particular layer extracted from a very large graph ($order of edges > 2^{10}$).
	\item{System Data Input for Scenario 2: }
	The individual explicitly defines a graph as an adjacency list and runs the algorithm on it. The \textit{Large Graphs} section of the web-tool can be used to visualize individual layers.
    \item{Input Data Types for Scenario 2:}
    A random graph can be generated by inputting the number of nodes and edges desired in the graph or a graph can be input as an adjacency list to the program.
	\item{System Data Output for Scenario 2: }
	The user obtains the file containing the core number and the files for each layer along with the files containing the information on the concentration of edges and computation time for each of them. These files can then be visualized.
    \item {Output Data Types for Scenario 2:}
    JSON files containing the data that can be visualized using the \textit{Large Graphs} section of the website. 
	\end{itemize}
\end{itemize}